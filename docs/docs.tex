% Created 2024-11-25 pon 19:07
% Intended LaTeX compiler: pdflatex
\documentclass[11pt]{article}
\usepackage[utf8]{inputenc}
\usepackage[T1]{fontenc}
\usepackage{graphicx}
\usepackage{longtable}
\usepackage{wrapfig}
\usepackage{rotating}
\usepackage[normalem]{ulem}
\usepackage{amsmath}
\usepackage{amssymb}
\usepackage{capt-of}
\usepackage{hyperref}
\usepackage[a4paper, margin=1.2in]{geometry}
\usepackage{algorithm}
\usepackage{algpseudocode}
\hypersetup{colorlinks=true,linkcolor=black}
\author{Piotr Jabłoński (325163) i Paweł Wysocki (325248)}
\date{Listopad 2024}
\title{POP - dokumentacja wstępna}
\hypersetup{
 pdfauthor={Piotr Jabłoński (325163) i Paweł Wysocki (325248)},
 pdftitle={Dokumentacja wstepna projektu z POP},
 pdfkeywords={},
 pdfsubject={},
 pdfcreator={Emacs 29.4 (Org mode 9.7.11)}, 
 pdflang={Polish}}
\begin{document}

\maketitle
\tableofcontents

\pagebreak
\section{Temat projektu}
\label{sec:org66d662c}
Celem naszego projektu jest rozwiązanie problemu plecakowego dla danych skorelowanych i nieskorelowanych używając algorytmu PBIL oraz porównanie jego działania z inną metodą. Należy dokonać dokładnej analizy statystycznej uzyskanych wyników. \\\\
Jako alternatywną metodę rozwiązania wybraliśmy \textbf{algorytm A*}.
\section{Opis problemu}
\label{sec:org4d991fb}
Problem plecakowy jest jednym z najpopularniejszych problemów optymalizacyjnych. 
Dysponujemy w nim listą $n$ przedmiotów, gdzie dla każdego z nich zdefiniowane są:\\
- wartość $p_i$,\\
- waga $w_i$.\\
Jest to problem maksymalizacyjny - należy wybrać z puli $k$ przedmiotów, dla których:
\begin{itemize}
    \item suma wartości $\sum_{i=1}^n y_i p_i$ jest największa,
    \item suma ich wag $\sum_{i=1}^n y_i w_i$ nie przekracza maksymalnej pojemności plecaka $W$,
    \item gdzie $y_i \in \{0,1\}$ - decyzja o spakowaniu przedmiotu.
\end{itemize}

\subsection{Funkcja celu}
\label{sec:orgfed3d26}
\textbf{Maksymalizowaną} funkcją celu w problemie plecakowym jest wspomniana wcześniej suma wartości:
$$
f(x) = \sum_{i=1}^n y_i \cdot p_i
$$
Rozwiązania niespełniające wymagania wagowego \textbf{zostaną obciążone funkcją kary w postaci sumy wartości wszystkich dostępnych przedmiotów}.
\subsection{Reprezentacja rozwiązania}
\label{sec:org7c490b7}
W klasycznym problemie plecakowym reprezentacją zadania jest wektorem bitów, gdzie $i$-ty bit informuje o decyzji, czy $i$-ty przedmiot został spakowany (efektywnie jest to reprezentacja wspomnianej we wcześniejszych wzorach zmiennej $y_i$). \\
Przykładowe reprezentacje mogą wyglądać następująco:
\begin{verbatim}
        01101011        spakowanie przedmiotów nr 2,3,5,7 i 8
        11111111        spakowanie wszystkich przedmiotów
        00000000        niespakowanie żadnego przedmiotu
\end{verbatim}

\pagebreak
\section{PBIL (Population-Based Incremental Learning)}
\label{sec:org5c31017}
Należy do rodziny algorytmów EDA (Estimation of Distribution Algorithm), znany jest ze swojej prostoty. Ta prostota wynika z faktu, że dystrybucja prawdopodobieństwa kolejnych bitów w chromosomie jest niezależna od punktu startowego. Dzięki temu algorytm można dowolnie modyfikować i optymalizować do konkretnego zadania.

Algorytm działa na zasadzie iteracyjnej poprawie populacji początkowej poprzez:
\begin{enumerate}
\item wygenerowanie M osobników z populacji P\textsuperscript{t}
\item ewualuacja i wybór najlepszych N osobników z M - podzbiór O\textsuperscript{t}
\item tworzenie nowej populacji P\textsuperscript{t+1} na podstawie populacji P\textsuperscript{t} oraz O\textsuperscript{t}
\item mutacja populacji P\textsuperscript{t+1}
\end{enumerate}
Kod w pseudo-pythonie przedstawiono poniżej:
\begin{algorithm}[h]
\caption{Algorytm PBIL}
\begin{algorithmic}
\State $initialize(p^0)$
\Comment{Inicjalizacja wektora prawdopodobieństw}
\State t = 0
\While {$!stop$}
\State $P^t = sample(p^t,M)$
\Comment{Generowanie $M$ osobników zgodnie z rozkładem $p^t$}
\State $O^t = select(P^t,N)$
\Comment{Selekcja $N$ najlepszych osobników z wygenerowanej populacji}
\State $p^{t+1} = update(O^t,p^t,a)$
\Comment{Aktualizacja wektora z użyciem wybranych osobników}
\State $p^{t+1} = mutate(p^{t+1})$
\Comment{Mutacja wektora prawdopodobieństw}
\State $t += 1$
\EndWhile
\end{algorithmic}
\end{algorithm}

Warunek końca to może być ilość iteracji lub satysfakcjonująco dobre rozwiązanie. Algorytm należy zatrzymać gdy wektor prawdopodobieństwa się ustabilizuje tzn. gdy z iteracji do iteracji występuje bardzo mały stopień poprawy.
\subsection{Reprezentacja zadania}
\label{sec:org0be5ed0}
W algorytmach ewolucyjnych rozwiązaniem zadania jest osobnik, a w każdej iteracji osobniki są mutowane indywidualnie. W przypadku algorytmów EDA sytuacja jest inna - optymalizuje się cały genotyp populacji na raz. Każda populacja jest reprezentowana jako dystrybucja prawdopodobieństwa. Algorytm PBIL należy do tej grupy i reprezentuje populację jako wektor prawdopodobieństw (p\textsubscript{i}):
\[
        \boldsymbol{p}^t = [p_1^t, p_2^t, \dots, p_n^t]
\]
Ten wektor będzie optymalizowany według \uline{\hyperref[sec:orgfed3d26]{funkcji celu}}.
\subsection{Funkcja \texttt{update}}
\label{sec:orga4c09fd}
Odpowiedzialna za aktualizację wektora prawdopodobieństw bazując na ilości jedynek w \textbf{x}, które znajdują się w N najlepszych rozwiązań. Wzór funkcji przedstawiono poniżej:
\[
        \boldsymbol{p}^{t+1}=(1-a) \cdot \boldsymbol{p}^t + a \cdot \frac{1}{N} \sum_{x \in O^t}x
\]
gdzie:
\begin{itemize}
\item a - learning rate
\item \textbf{x} - binarny wektor opisany w \hyperref[sec:org7c490b7]{\uline{reprezentacji rozwiązania}}
\end{itemize}
\section{A*}
\label{sec:orgd1b064a}
A* jest przykładem algorytmu wyczerpującego przeszukiwania przestrzeni. Jest to algorytm zupełny i optymalny, tym sensie, że jeżeli optymalne rozwiązanie istnieje, to zostanie znalezione. Jest to typowe narzędzie do rozwiązywania problemów drzewiastych, takich jak problem plecakowy. Uznaliśmy że cecha optymalności tego algorytmu pozwoli na ciekawą analizę wyników w porównaniu do alg. \hyperref[sec:org5c31017]{\uline{PBIL}}.
\subsection{Reprezentacja rozwiązania w A*}
\label{sec:org4ecd5a9}
Nieco różni się od \hyperref[sec:org7c490b7]{pierwszego podejścia} tym, że wektor musi zostać rozszerzony o znak ?, który oznacza \textbf{brak decyzji}. Algorytm działa na zasadzie tworzenia "ścieżki", więc tylko węzeł końcowy będzie reprezentował kompletne rozwiązanie, tzn. informacje o wpakowaniu każdego przedmiotu.
\begin{verbatim}
        ????????        punkt startowy algorytmu
        01??????        wpakowanie 2; reszta nieznana - poziom 2 drzewa
        01101???        wpakowanie 2,3,5; reszta nieznana
        01101110        wpakowanie 2,3,5,6,7; węzeł końcowy
\end{verbatim}
\subsection{Funkcja zysku i heurystyczna}
\label{sec:orgd7b7848}
W przypadku problemu plecakowego A* działa na zasadzie \textbf{maksymalizacji} funkcji f, która jest definiowana przez:
\[
        f(x) = g(x) + h(x)
\]
gdzie:
\begin{itemize}
\item g(x) - funkcja zysku
\item h(x) - funkcja heurystyczna
\end{itemize}
\subsubsection{Funkcja zysku}
\label{sec:orgf97f734}
Najbardziej sensownym podejściem będzie zsumowanie wartości przedmiotów \textbf{w plecaku}
\[
        g(x) = \sum_{i:x_i=1}^n{v_i}
\]
Jeżeli wpakujemy dodatkowy przedmiot: rozwiązanie zwiększy swoją sumaryczną wartość się o wartość wsadzonego przedmiotu.
\subsubsection{Funkcja heurystyczna}
\label{sec:org8963b0e}
Musi być:
\begin{itemize}
\item dopuszczalna: g(x) + h(x) \(\ge\) g(x\textsubscript{t})
\item monotoniczna: g(x\textsubscript{j}) + h(x\textsubscript{j}) \(\le\) g(x\textsubscript{i}) + h(x\textsubscript{i})
\end{itemize}
Dla problemu plecakowego można użyć funkcji heurystycznej postaci:
\[
        h(x) = \sum_{i:x_i=?}^n{y_i \cdot p_i}
\]
gdzie
\begin{itemize}
\item i:x\textsubscript{i}=? - indeksy w x dla przedmiotów o statusie ?
\item y\textsubscript{i} to zmienna ułamkowa, definiowana przez równość
\end{itemize}
\[
        \sum_{i:x_i=?}^n{y_i \cdot w_i} = W - \sum_{i=1}^n{x_i \cdot w_i}
\]
\section{Plan eksperymentów}
\label{sec:org78ee836}
Aby przeprowadzić dokładną analizę statystyczną porównującą efektywność obu metod, wymagane jest odpowiednie środowisko testowe. Do tego zadania wybraliśmy 3 różne problemy plecakowe, które różnią się od siebie poziomem skorelowania danych:
\begin{enumerate}
\item Dane nieskorelowane
\item Dane średnio skorelowane
\item Dane mocno skorelowane
\end{enumerate}
Testy przeprowadzimy dla różnej maksymalnej ilości przedmiotów, wagi przedmiotu oraz pojemnośic plecaka. Początkowo przyjmujemy n = 100, v = 10, W = 100.

Takie podejście pozwoli na sprawdzenie efektywności metody \uline{\hyperref[sec:org5c31017]{PBIL}} w porównaniu do metody \hyperref[sec:orgd1b064a]{\uline{A*}}, które zawsze znajdzie optymalne rozwiązanie. Każdy eksperyment będzie uruchomiony wielokrotnie, a wyniki uśrednione z dokładnością do drugiego miejsca po przecinku.

Parametr, który nas interesuje podczas testowania efektywności najbardziej to \textbf{maksymalna nagrana wartość funkcji celu} oraz \textbf{ilość obliczeń funkcji celu}. Do analizy wizualnej posłużą nam wykresy, które mapują wartości funkcji celu na ilość iteracji algorytmu oraz tabelki ze średnimi wartościami i standardowym odchyleniem. Informacje, które będziemy zbierać:
\begin{verbatim}
{
        maksymalna_wartość_funkcji_celu,        # najlepszego osobnika
        końcowa_wartość_funkcji_celu,           #
        średnia_wartość_funkcji_celu,           # całej populacji
        odchylenie_standardowe_funkcji_celu     #
}
\end{verbatim}
\end{document}
