% Created 2024-11-25 pon 19:07
% Intended LaTeX compiler: pdflatex
\documentclass[11pt]{article}
\usepackage[utf8]{inputenc}
\usepackage[T1]{fontenc}
\usepackage{graphicx}
\usepackage{longtable}
\usepackage{wrapfig}
\usepackage{rotating}
\usepackage[normalem]{ulem}
\usepackage{amsmath}
\usepackage{amssymb}
\usepackage{capt-of}
\usepackage{hyperref}
\usepackage[a4paper, margin=1.2in]{geometry}
\usepackage{algorithm}
\usepackage{algpseudocode}
\setlength{\parindent}{0pt}
\hypersetup{colorlinks=true,linkcolor=black}
\author{Piotr Jabłoński (325163) i Paweł Wysocki (325248)}
\date{Listopad 2024}
\title{POP - dokumentacja wstępna}
\hypersetup{
 pdfauthor={Piotr Jabłoński (325163) i Paweł Wysocki (325248)},
 pdftitle={Dokumentacja wstepna projektu z POP},
 pdfkeywords={},
 pdfsubject={},
 pdfcreator={Emacs 29.4 (Org mode 9.7.11)}, 
 pdflang={Polish}}
\begin{document}

\maketitle
\tableofcontents

\pagebreak
\section{Temat projektu}
\label{sec:org66d662c}
Celem naszego projektu jest rozwiązanie problemu plecakowego dla danych skorelowanych i nieskorelowanych używając algorytmu PBIL oraz porównanie jego działania z inną metodą. Należy dokonać dokładnej analizy statystycznej uzyskanych wyników. \\\\
Jako alternatywną metodę rozwiązania wybraliśmy \textbf{algorytm A*}.
\section{Opis problemu}
\label{sec:org4d991fb}
Problem plecakowy jest jednym z najpopularniejszych problemów optymalizacyjnych. 
Dysponujemy w nim listą $n$ przedmiotów, gdzie dla każdego z nich zdefiniowane są:\\
- wartość $p_i$,\\
- waga $w_i$.\\
Jest to problem maksymalizacyjny - należy wybrać z puli $k$ przedmiotów, dla których:
\begin{itemize}
    \item suma wartości $\sum_{i=1}^n x_i p_i$ jest największa,
    \item suma ich wag $\sum_{i=1}^n x_i w_i$ nie przekracza maksymalnej pojemności plecaka $W$,
    \item gdzie $x_i \in \{0,1\}$ - decyzja o spakowaniu przedmiotu.
\end{itemize}

\subsection{Funkcja celu}
\label{sec:orgfed3d26}
\textbf{Maksymalizowaną} funkcją celu w problemie plecakowym jest wspomniana wcześniej suma wartości:
$$
f(x) = \sum_{i=1}^n x_i \cdot p_i
$$
Rozwiązania niespełniające wymagania wagowego \textbf{zostaną obciążone funkcją kary w postaci sumy wartości wszystkich dostępnych przedmiotów}.
\subsection{Reprezentacja rozwiązania}
\label{sec:org7c490b7}
W klasycznym problemie plecakowym reprezentacją zadania jest wektorem bitów, gdzie $i$-ty bit informuje o decyzji, czy $i$-ty przedmiot został spakowany (efektywnie jest to reprezentacja wspomnianej we wcześniejszych wzorach zmiennej $x_i$). \\
Przykładowe reprezentacje mogą wyglądać następująco:
\begin{verbatim}
        01101011        spakowanie przedmiotów nr 2,3,5,7 i 8
        11111111        spakowanie wszystkich przedmiotów
        00000000        niespakowanie żadnego przedmiotu
\end{verbatim}

\pagebreak
\section{Algorytmy}
\subsection{PBIL (Population-Based Incremental Learning)}
\label{sec:org5c31017}
Algorytm PBIL należy do rodziny algorytmów EDA (Estimation of Distribution Algorithm), polegających na próbkowaniu i oszacowywaniu rozkładu prawdopodobieństwa wybranych rozwiązań zawartych w populacji, zamiast klasycznej ewolucji przez losowe krzyżowanie osobników. Dystrybucja prawdopodobieństwa kolejnych bitów w chromosomie jest niezależna od punktu startowego, dzięki temu algorytm można swobodnie modyfikować i optymalizować pod kątem własnych potrzeb.\\\\
Algorytm działa na zasadzie iteracyjnej poprawy populacji początkowej poprzez:
\begin{enumerate}
\item generację $M$ osobników z populacji $P^t$
\item ewaluację i wybór najlepszych $N$ wygenerowanych osobników - podzbiór $O^t$
\item tworzenie nowej populacji $P^{t+1}$ na podstawie populacji $P^t$ oraz $O^t$
\item mutację populacji $P^{t+1}$
\end{enumerate}

\subsubsection{Pseudokod algorytmu}
\begin{algorithm}[h]
\caption{Algorytm PBIL}
\begin{algorithmic}
\State $initialize(p^0)$
\Comment{Inicjalizacja wektora prawdopodobieństw}
\State t = 0
\While {$!stop$}
\State $P^t = sample(p^t,M)$
\Comment{Generowanie $M$ osobników zgodnie z rozkładem $p^t$}
\State $O^t = select(P^t,N)$
\Comment{Selekcja $N$ najlepszych osobników z wygenerowanej populacji}
\State $p^{t+1} = update(O^t,p^t,a)$
\Comment{Aktualizacja wektora z użyciem wybranych osobników}
\State $p^{t+1} = mutate(p^{t+1})$
\Comment{Mutacja wektora prawdopodobieństw}
\State $t += 1$
\EndWhile
\end{algorithmic}
\end{algorithm}

Warunkiem stopu będzie osiągnięcie limitu iteracji, osiągnięcie satysfakcjonującego rozwiązania lub ustabilizowanie się wektora prawdopodobieństwa (bardzo mały stopień poprawy jakości dla kolejnych generacji).
\subsubsection{Reprezentacja zadania}
\label{sec:org0be5ed0}
W klasycznych algorytmach ewolucyjnych rozwiązaniem zadania jest konkretny osobnik, a w każdej iteracji osobniki są mutowane indywidualnie. Natomiast w przypadku algorytmów EDA optymalizuje się cały genotyp populacji na raz. Każda populacja jest reprezentowana jako dystrybucja prawdopodobieństwa. Algorytm PBIL reprezentuje populację jako wektor prawdopodobieństw ($p^t$):
$$
p^t = [p^t_1,p^t_2,\dots,p^t_n]
$$
Wektor ten będzie optymalizowany według \uline{\hyperref[sec:orgfed3d26]{funkcji celu}}. Jest on inicjowany wartościami $0.5$, dzięki czemu początkowy rozkład jest równomierny i poszukiwanie rozwiązania nie jest obciążone.

\subsubsection{Funkcja \texttt{update}}
\label{sec:orga4c09fd}
Funkcja ta jest odpowiedzialna za aktualizację wektora prawdopodobieństw na podstawie częstotliwości występowania jedynek dla każdego genu wśród osobników ze zbioru $O^t$. Wzór funkcji przedstawiono poniżej:
$$
p^{t+1} = (1-a) \cdot p^t + a \cdot \frac{1}{N} \sum_{x \in O^t} x
$$
gdzie:
\begin{itemize}
\item $a$ - learning rate
\item \textbf{x} - pojedynczy osobnik, binarny wektor opisany w \hyperref[sec:org7c490b7]{\uline{reprezentacji rozwiązania}}
\end{itemize}

\subsubsection{Hiperparametry}
Zadanie wymaga ustalenia wartości następujących parametrów:
\begin{itemize}
\item $M$ - rozmiar generowanej populacji
\item $N$ - liczba najlepszych osobników wybieranych z wygenerowanej populacji
\item $a$ - learning rate algorytmu
\end{itemize}

\subsection{Algorytm A*}
\label{sec:orgd1b064a}
Algorytm A* jest przykładem algorytmu wyczerpującego przeszukiwania przestrzeni. Jest to algorytm zupełny i optymalny, czyli gwarantuje znalezienie optymalnego rozwiązania, jeżeli tylko takowe istnieje. Jest on powszechnie wykorzystywany do rozwiązywania problemów reprezentowanych przez strukturę drzewiastą. Jego gwarancja znalezienia optymalnego rozwiązania czyni go bardzo ciekawym w kontekście porównania wyników z algorytmem \hyperref[sec:org5c31017]{\uline{PBIL}}.

\subsubsection{Reprezentacja rozwiązania w A*}
\label{sec:org4ecd5a9}
Algorytm A* wymaga reprezentacji przestrzeni rozwiązań w strukturze drzewiastej, więc reprezentację z \hyperref[sec:org7c490b7]{\uline{pierwszego podejścia}} należy rozszerzyć o dodatkową wartość $?$, która oznacza, że \textbf{nie podjęto jeszcze decyzji o tym przedmiocie}. Na każdym następnym poziomie drzewa znajdujący się najbardziej na lewo $?$ zostaje zastąpiony wartością $0$ (przedmiot nie został spakowany) lub $1$ (przedmiot spakowano). Węzły terminalne to takie, które nie spełniają założeń zadania niezależnie od tego, jakie decyzje zostaną jeszcze podjęte (przekroczenie wagi), lub takie, dla których zostały podjęte wszystkie decyzje.\\
Przykładowe reprezentacje wyglądają następująco:
\begin{verbatim}
        ????????        punkt startowy algorytmu, nie podjęto żadnej decyzji
        01??????        spakowanie 2; reszta nieznana - poziom 2 drzewa
        01101???        spakowanie 2,3,5; reszta nieznana
        01101110        spakowanie 2,3,5,6,7; węzeł końcowy
\end{verbatim}

\subsubsection{Funkcja celu}
\label{sec:orgd7b7848}
W przypadku algorytmu A*, \textbf{maksymalizowana} funkcji celu ma postać:
$$
f(x) = g(x) + h(x)
$$
gdzie:
\begin{itemize}
\item $g(x)$ - funkcja zysku
\item $h(x)$ - funkcja heurystyczna
\end{itemize}

\subsubsection{Funkcja zysku}
\label{sec:orgf97f734}

Funkcja zysku ma postać:
$$
g(x) = \sum_{i = 1}^n x_i \cdot p_i
$$
Jest ona trywialna - stanowi sumę wartości spakowanych przedmiotów.
\subsubsection{Funkcja heurystyczna}
\label{sec:org8963b0e}
Musi być:
\begin{itemize}
\item dopuszczalna: $g(x) + h(x) \ge g(x_t)$ - musi cechować się tzw. "nadmiernym optymizmem", czyli przeszacowywać możliwy zysk.
\item monotoniczna: $g(x_j)+h(x_j) \le g(x_i) + h(x_i)$ - błąd oszacowania musi maleć wraz ze zbliżaniem się do rozwiązania.
\end{itemize}
Zaproponowana przez nas funkcja heurystyczna to:
$$
h(x) = \sum_{i:x_i=?}^n y_i \cdot p_i
$$
gdzie:
\begin{itemize}
\item $i: x_i = ?$ - indeksy w $x$ dla przedmiotów, dla których nie podjęto jeszcze decyzji
\item $y_i \in [0;1]$ - zmienna ułamkowa (\textbf{w funkcji heurystycznej dopuszczamy pakowanie części przedmiotów}) wyznaczana wg wzoru:
\end{itemize}
$$
\sum_{i:x_i=?}^n y_i \cdot w_i = W - \sum_{i=1}^n x_i \cdot w_i
$$
Maksymalizacja funkcji celu sprawia, że w funkcji heurystycznej będą premiowane przedmioty o najwyższych współczynnikach $\frac{p_i}{w_i}$. Funkcja ta jest dopuszczalna i monotoniczna, więc nadaje się na funkcję heurystyczną do naszego zadania.

\section{Plan eksperymentów}
\label{sec:org78ee836}
W celu przeprowadzenia dokładnej analizy statystycznej porównującej efektywność obu metod, wymagane jest odpowiednie środowisko testowe. Do tego zadania wybraliśmy 3 różne problemy plecakowe, które różnią się od siebie poziomem skorelowania danych:
\begin{enumerate}
\item Dane nieskorelowane
\item Dane średnio skorelowane
\item Dane mocno skorelowane
\end{enumerate}
Testy przeprowadzimy dla różnej maksymalnej ilości przedmiotów, wagi przedmiotu oraz pojemności plecaka. Początkowo przyjmujemy $n = 100,\ v = 10,\ W = 100$. Takie podejście pozwoli na sprawdzenie efektywności metody \uline{\hyperref[sec:org5c31017]{PBIL}} w porównaniu do \hyperref[sec:orgd1b064a]{\uline{algorytmu A*}}, który zawsze znajdzie optymalne rozwiązanie. 

\subsection{Badane czynniki}
W naszej analizie skupimy się na następujących czynnikach:
\begin{itemize}
\item jakość wyznaczonego rozwiązania
\item czas dojścia do optymalnego rozwiązania
\item wpływ parametrów na działanie algorytmu
\end{itemize}

Wszelkie eksperymenty zostaną wykonane wielokrotnie w celu uwiarygodnienia wyników i eliminacji wpływu czynników zewnętrznych (np. dodatkowe obciążenie w tle podczas mierzenia czasu wyznaczania rozwiązań). Dodatkowo, skorzystamy ze standardowych metryk analizy statystycznej, tj. wartości maksymalnej, średniej i odchylenia standardowego dla każdego czynnika. Wszelkie zebrane dane zostaną przedstawione na odpowiednich wykresach i szczegółowo przeanalizowane.

\end{document}
